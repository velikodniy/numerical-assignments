\section{Численное интегрирование}

\Aim{Исследовать методы численного интегрирования с помощью
  электронных вычислительных машин.}

Оценим погрешность на одном из подотрезков.
\[
\Delta I_i = \int_{x_i}^{x_i+\Delta x} (f(x) - f(x_i))\,dx.
\]

Выполнив замену $t = x - x_i$ получим
\[
\Delta I_i = \int_0^{\Delta x} (f(t + x_i) - f(x_i))\, dt.
\]

Разложим функцию $f$ в ряд Тейлора в окрестности точки $x_i$:
\[
    f(t + x_i) = f(x_i) + f'(x_i)t + f''(x_i)t^2 + O(t^3).
\]

Подставляя в интеграл, получаем
\begin{multline*}
\Delta I_i = \int_0^{\Delta x} \left(f(x_i) + f'(x_i)t + f''(x_i)t^2 + O(t^3) - f(x_i)\right)\,dt =\\
=\int_0^{\Delta x}\left(f'(x_i)t + f''(x_i)t^2 + O(t^3)\right)\,dt=\\
=\frac12 f'(x_i) \bigl. t^2 \bigr|_{0}^{\Delta x} + \frac13 f''(x) \bigl. t^3 \bigr|_{0}^{\Delta x} +  \bigl.O(t^4) \bigr|_{0}^{\Delta x}=\\
=\frac12 f'(x_i) \Delta x^2 + \frac13 f''(x) \Delta x^3 + O(\Delta x^4).
\end{multline*}

Так как $f'(x_i)$ и $f''(x_i)$ — константы, их можно внести под знак O. Тогда
\[
\Delta I_i = O(\Delta x^2).
\]

Погрешность на всём отрезке $[a, b]$ будет складываться из погрешностей отдельных подотрезков
\[
\Delta I = \sum_{i=0}^{N-1}\Delta I_i = \sum_{i=0}^{N-1}O(\Delta x^2) = N O(\Delta x^2) = O((N \Delta x) \Delta x) = O((b-a)\Delta x).
\]

Так как $b-a$ — константа, то
\[
\Delta I = O(\Delta x).
\]