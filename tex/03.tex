\section{Решение нелинейных уравнений}

\noindent \textbf{Цель:} Исследовать методы приближённого решения
нелинейных уравнений с одним неизвестным при помощи электронных вычислительных
машин.

\medskip{}



\subsubsection{Общий вид уравнений}
\begin{defn}
Уравнениями называются равенства, содержащие один или несколько неизвестных
числовых параметров.
\end{defn}

\begin{defn}
Решение уравнения — это такой набор параметров, который превращает
равенство в тождество.
\end{defn}
У уравнения может быть одно, несколько решений либо не быть решений
вовсе.

Существует большое количество классов уравнений для которых известны
методы решения. Например, линейные, квадратические, многие тригонометрические.
Однако, не для всех уравнений решение можно записать через элементарные
функции. Так, по теореме Абеля—Руффини, алгебраические уравнения начиная
с 4-й степени неразрешимы в радикалах.

Многие просто выглядящие уравнения решаются аналитически довольно
сложно. Например,
\[
\text{\ensuremath{\cos}}x=x.
\]
Поэтому для решения часто оправдано применение численных методов.

В общем виде уравнение с одним неизвестным можно записать в виде:
\[
f(x)=0.
\]
Любое уравнение можно привести к такому виду рассмотрев разность правой
и левой частей равенства. Иногда рассматривают уравнение в виде:
\[
F(x)=x.
\]
Очевидно, обе формы эквивалентны и равноправны.

Как правило, методы решения алгебраических уравнений являются итерационными.
То есть, они не сразу дают ответ, а последовательно уточняют его,
пока не будет достигнута необходимая точность. Обычно в качестве критерия
прекращения итераций используют неравнество:
\[
|f(x)|<\varepsilon.
\]


Обычно методы позволяют найти лишь один корень на заданно отрезке.
Поэтому становится важной задача отделения корней, то есть поиска
интервалов, на которых распологается лишь один корень. Это можно сделать
приближённо с помощью графика функции.


\subsubsection{Метод последовательных приближений}

Метод последовательных приближений — это один из наиболее простых,
но в то же время наиболее универсальных методов, применяющихся в самых
разных разделах численных методов, в том числе и для решения алгебраических
уравнений. Он описрается на теорему о сжимающем отображении.
\begin{defn}
Отображение $y=F(x)$ называется сжимающим, если $\forall x_{1},x_{2}$
$\exists q\in[0,1]$, что
\[
|F(x_{1})-F(x_{2})|\leqslant q|x_{1}-x_{2}|.
\]

\end{defn}
То есть, отображение называется сжимающим, если расстояние между образами
двух любых точек всегда меньше, чем расстояние между ними самими.

\emph{Рисунок}

Из определения можно вывести признак сжимающего отображения для непрерывной
функции $F$. В самом деле, из определения следует
\[
\frac{|F(x_{1})-F(x_{2})|}{|x_{1}-x_{2}|}\leqslant q<1,
\]
\[
|\mathrm{tg}\alpha|<1,
\]
где $\alpha$ — угол наклона секущей к графику, проходяящей через
выбранные точки. Учитывая геометрический смысл производной, можно
сделать вывод, что если
\[
|F'(x)|<1,
\]
отображаения $F$ будет сжимающим.
\begin{thm}
Последовательность точек $\{x_{k}\},$ где $k=0,1,2,\dots$, порождённая
итерационным процессом 
\[
x_{k+1}=F(x_{k})
\]
 сходится к решению $x^{*}$ уравнения 
\[
x=F(x),
\]
если $F$ — сжимающее отображение. При этом
\[
|x^{*}-x_{k}|\leqslant\frac{q}{1-q}|x_{k}-x_{k-1}|.
\]

\end{thm}
Эта теорема опирается на другую теорему, согласно которой сжимающее
отображение имеет одну и только одну неподвижную точку.

Таким образом, из теоремы следует алгоритм решения уравнения $F(x)=x$. 
\begin{enumerate}
\item Выбираем произвольную точку $x_{0},\, k=0$.
\item Находим следующее приближение $x_{k+1}=F(x_{k})$.
\item Если не достигнута необходимая точность, то увеличиваем $k$ и переходим
к п. 2.
\item Последняя найденная точка — искомое численное решение.
\end{enumerate}

\subsubsection{Метод дихотомии}

Этот метод основан на широко распространённом методе «деления пополам»,
когда искомое значение берётся в »вилку». Так люди ищут слово в словаре,
артиллеристы пристреливают цели и т.~д. Подобный приём можно применить
и для поиска решений уравнений.

Метод дихотомии (от греч «дихэ» — надвое, и «томэ» — деление) применяется,
если функция непрерывна на отрезке $[a,b]$ и монотонна, а также 
\[
f(a)>0,\, f(b)<0.
\]
При этом предполагается существование на отрезке единственного корня.

Приведённые неравенства не уменьшают общность метода, так как любую
функцию с разными знаками на концах отрезка можно «перевернуть», умножив
на $-1,$ не изменив при этом корни.

Суть метода такова. Исходный отрезок делится на две части некотрой
точкой $c=\frac{a+b}{2}$. При этом, очевидно, единственный корень
попадает только в один из подотрезков либо совпадает с точкой $c$
(последнее легко проверяется).

\emph{Рисунок}

Определить, в какой из подотрезков попал корень можно по знаку функции
в точке $c$. Если $f(c)<0$ — корень слева, иначе — справа.

Отрезок с корнем берётся в качестве исходного и действия повторяются
до тех пор, пока полученный отрезок не станет достаточно мал. В качестве
корня можно взять любую его точку. Длина полученного малого отрезка
и будет определять погрешность найденного корня.

Если длина полученного отрезка после $n$ итераций оказалась меньше
$\varepsilon$, то, учитывая, что на каждом шаге длины сокращаются
вдвое, получаем неравенство
\[
\frac{b-a}{2^{n}}<\varepsilon,
\]
откуда
\[
n>\log_{2}\frac{b-a}{\varepsilon}.
\]


Получили выражение для определения числа итераций, требуемого для
достижения необходимой точности.


\subsubsection{Метод Ньютона}

За универсальность метода дихотомии приходится платить сравнительно
медленной скоростью сходимости. Существуют и более «быстрые» методы.
Одним из которых является метод Ньютона (разработан И. Ньютоном в
1669 г.) и его модификации.

Будем решать уравнение вида 
\[
f(x)=0.
\]


Пусть есть некоторый итерационный процесс и текущее приближение $x$.
Неизвестный корень обозначим $x^{*}$. Предположим, что найдя приближение,
мы ошиблись в нахождении корня на некоторую (тоже неизвестную) величину
$\Delta x$, тогда
\[
x^{*}=x+\Delta x.
\]


Так как $x^{*}$ — корень, то
\[
f(x^{*})=0.
\]


Рассмотрим разложение в ряд Тейлора.
\[
0=f(x^{*})=f(x+\Delta x)=f(x)+f'(x)\Delta x+O(\Delta x^{2}).
\]


Пренебрегая остаточным членом, получаем:
\[
0\approx f(x)+f'(x)\Delta x.
\]


Отсюда
\[
x^{*}=x+\Delta x\approx x-\frac{f(x)}{f'(x)}.
\]


Так как мы пренебрегали членами высоких порядков, ответ получается
неточным. поэтому полученное значение можно использовать вновь в качестве
текущего приближение. Вычисления повторяются до тех пор, пока не будет
достигнута требуемая точность.

Таким образом, получаем итерационный процесс:
\[
x_{k+1}=x_{k}-\frac{f(x_{k})}{f'(x_{k})}.
\]


Заметим, что этот итерационный процесс представляет собой решение
уравнения
\[
x=x-\frac{f(x)}{f'(x)},
\]
имеющего те же корни, что и
\[
f(x)=0,
\]
методом последовательных приближений.
\begin{thm}
Пусть существуют первые две производные функции $f(x)$ и при этом
\[
|f'(x)|\leqslant C_{1},\,|f''(x)|\leqslant C_{2},
\]
тогда если
\[
C_{1}^{2}C_{2}|f(x_{0})|\leqslant q<1,
\]
то метод Ньютона сходится, причём с квадратичной скоростью.
\end{thm}
Геометрический смысл метода Ньютона заключается в последовательном
нахождении касательных к графику и точек их пересечения с осью $Ox$.
Поэтому этот метод также называют методом касательных.

\emph{Рисунок}

Метод Ньютона создаёт определённые трудности при программировании,
так как требуется знать производную функции $f$. Поэтому для вычисления
производной часто тоже применяют численные методы либо используют
различные приёмы.

Например, часто используется т.н. модифицированный метод Ньютона:
\[
x_{k+1}=x_{k}-\frac{f(x_{k})}{f'(x_{0})},
\]
который, однако, имеет линейную сходимость.

Другой подход состоит в замене касательных секущими. Пусть $x_{k}=x_{k-1}-\Delta x,$
тогда по определению
\[
f'(x_{k})=\lim_{\Delta x\rightarrow0}\frac{f(x_{k}+\Delta x)-f(x_{k})}{\Delta x}=\lim_{x_{k-1}-x_{k}\rightarrow0}\frac{f(x_{k-1})-f(x_{k})}{x_{k-1}-x_{k}}\approx\frac{f(x_{k})-f(x_{k-1})}{x_{k}-x_{k-1}}.
\]


В этом случае получаем метод хорд:
\[
x_{k+1}=x_{k}-\frac{f(x_{k})(x_{k}-x_{k-1})}{f(x_{k})-f(x_{k-1})}.
\]


Если в приближённом выражении для производной зафиксировать одну из
точек, получаем метод секущих (иногда называется методом ложного положения
или regula falsi):
\[
x_{k+1}=x_{k}-\frac{f(x_{k})(x_{k}-x_{0})}{f(x_{k})-f(x_{0})}.
\]


Это приближение несколько снижает скорость сходимости, но избавляет
от необхоимости хранить предыдущие вычисленные значения.

Задачи:

Написать программу для приближённого решения нелинейных уравнений,
заданных функцией. При необходимости модифицировать уравнение, чтобы
оно подходило под ограничения метода. Сравнить результат с точным,
вычислить абсолютную и относительную погрешности. Определить количество
шагов, потребовавшихся для поиска решения.

Необходимо запрограммировать и исследовать следующие методы:

метод последовательных приближений; метод дихотомии; метод хорд.

При желании можно также реализовать:

метод Ньютона (касательных); метод секущих.

В результатах должно быть представлено:

решаемое уравнение; параметры метода (начальное приближение или отрезок,
на котором ищется корень); задаваемые погрешности для критерия прекращения
расчётов; полученное приближённое решение; количество итераций, потребовавшихся
для получения результата.

В заключительной части нужно сделать выводы, дающие ответ на следующие
вопросы:

Насколько рассмотренные методы подходят для приближённого решения
уравнений? Какой из методов потребовал наименьшее число итераций для
поиска решения? Как уменьшить требуемое число итераций? Потребовалось
ли преобразовывать уравнение для того, чтобы применить к нему методы?
В каких ситуациях это требуется делать?

\newpage{}


\section{Решение задач линейной алгебры}

\noindent \textbf{Цель:} Исследовать прямые и итерационные методы
решения систем линейных алгебраических уравнений и других задач линейной
алгебры при помощи электронных вычислительных машин.

