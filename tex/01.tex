\Lab{Вычисление значений функций}

\Aim{Изучить элементы теории погрешностей и исследовать методы
  приближённого вычисления значений функций при помощи электронных
  вычислительных машин.}

\Theory

\Section{Погрешности}

Любое отличие от верного ответа — это ошибка, вносящая в результат
некоторую погрешность. При решении практических задач с использованием
численных методов, ошибки неизбежны. Причём ошибки не обязательно
связаны с некачественным программным обеспечением или сбоями
оборудования.  Часто причиной является неправильный выбор метода или
вычисления без учёта ограниченных возможностей вычислительной техники.

Выделяют следующие основные виды погрешностей.

\begin{itemize}
\item\Term{Погрешность!неустранимая}{Неустранимая погрешность} —
  математическое описание задачи неточно, т. е. либо неточны входные
  данные, либо неточна сама модель.
\item\Term{Погрешность!метода}{Погрешность метода} — метод не является
  точным, при его разработке были сделаны какие-то допущения,
  приводящие к ошибке.
\item\Term{Погрешность!вычислительная}{Вычислительная погрешность} —
  погрешности и ошибки в ходе расчётов.
\end{itemize}

\Section{Виды погрешностей}

Сам факт наличия погрешности мало что говорит о качестве решения.
Погрешность может быть как приемлемой, так и недопустимой. Для того,
чтобы принимать решения о допустимости погрешностей необходимо уметь
оценивать их количественно. Качество результатов принято
характеризовать \Term{Погрешность!абсолютная}{абсолютной} и
\Term{Погрешность!относительная}{относительной погрешностями}.

Пусть $x$ — точное значение, а $\tilde{x}$ — приближённое. Тогда
погрешности вычисляются следующим образом.
\begin{itemize}
\item Абсолютная погрешность:
  \[
  \boxed{\Delta x = \left|x-\tilde{x}\right|.}
  \]
\item Относительная погрешность
  \[
  \boxed{\delta x = \left|\frac{x-\tilde{x}}{\tilde{x}}\right|.}
  \]
\end{itemize}

\begin{example}
  При помощи некоторого численного метода решалось уравнение $x-3=0$.
  Очевидно, точное решение этого уравнения — $x=3$. При этом в
  качестве результата было получено значение $\tilde{x}=3{,}017$.

  В этом случае погрешности будут равны
  \begin{eqnarray*}
    \Delta x & = & \left|3 - 3{,}017\right| = 0{,}017 = 1{,}7\times10^{-2},\\
    \delta x & = & \left|\frac{3-3{,}017}{3{,}017}\right| \approx
      0{,}00563 = 5{,}663\times10^{-3}.
  \end{eqnarray*}
\end{example}

\Section{Свойства погрешностей}

При оценке погрешностей можно пользоваться следующими свойствами:
\begin{eqnarray*}
  \Delta(x_1 + x_2) & \leqslant & \Delta x_1 + \Delta x_2,\\
  \Delta(x_1 - x_2) & \leqslant & \Delta x_1 + \Delta x_2,\\
  \Delta(x_1 x_2)   & \leqslant & x_1 \Delta x_2 + x_2 \Delta x_1.
\end{eqnarray*}

При более сложных вычислениях формулы существенно усложняются.

Как видно, при арифметических операциях над величинами погрешность
накапливается.

При сравнении погрешностей, полученных в результате решения различных
задач, используют относительную погрешность, так как она учитывает
масштаб величин. В самом деле, если производились вычисления над
малыми величинами, то и модуль их разности — абсолютная погрешность —
будет малой величиной, даже если величины значительно различаются.

\Section{Символы Ландау}

Чтобы вычислить погрешность, необходимо знать точное значение
величины, что в большинстве случаев невозможно. Однако, погрешность
часто можно оценить исходя из свойств применямого метода или алгоритма
вычислений.

Для записи оценок часто используют так называемые \Term{Символы
  Ландау}{символы Ландау}. Это специальные математические обозначения,
позволяющие скрыть коэффициент пропорциональности, соответствующий
быстродействию, в выражениях для сложности.

Чаще всего используют символ $O$ (читается как \Term{O большое}{«О
    большое»}).

\begin{defn}
  Говорят, что $f(x)=O(g(x))$, если существуют такие константы
  $C\in\mathbb{R}$ и $x_{0}\in\mathbb{R}$, что для любого $x>x_{0}$
  выполняется неравенство $|f(x)|\leqslant|Cg(x)|$.
\end{defn}

То есть, приведённое обозначение говорит о том, что $f$ асимптотически
ограничена сверху функцией $g$ с точностью до постоянного
множителя. Иными словами, символ $O$ «скрывает» константу и медленно
растущие слагаемые.

\begin{example}
  Например, $2x^2+7x+5 = O(x^2)$ и $5x\log_2 x + 9x + 4 = O(x\log_2
  x)$. Здесь символ $O$ скрыл константу и член меньшего порядка.
\end{example}

\begin{example}
  Так называемый метод трапеций вычисления определённых интегралов
  имеет параметр $h$, определяющий точность вычислений. При этом
  погрешность метода равна $O(h^{2})$. $O$ — это не функция, а
  специальное математическое обозначение.
  
  По такой записи невозможно установить, чему равна погрешность для
  конкретных значений $h$. Однако, если строить график зависимости
  погрешности от этого параметра, то он будет вести себя
  приблизительно как парабола — график функции $h^2$.

  Отсюда, в частности, следует, что если уменьшить $h$ в 10 раз, то
  погрешность уменьшится приблизительно в $10^{2}=100$ раз.
\end{example}

Если погрешность метода с некоторым малым параметром $h$ имеет вид
$O(h^{n}),$ то величина $n$ называется \Term{Порядок
  точности}{порядком точности} численного метода. Чем она выше, тем,
как правило, «лучше» метод.

Очевидно, что величину погрешности можно уменьшить, сведя количество
выполняемых вычислений к минимуму, так как каждое арифметическое
действие приводит к увеличению общей погрешности. Поэтому
рекомендуется при расчётах использовать экономные методы вычислений.

\Section{Схема Горнера}

Для уменьшения погрешности при вычислении значений многочленов можно
применять \Term{Схема Горнера}{схему Горнера}:

\begin{multline*}
    a_0 + a_1 x + a_2 x^2 + a_3 x^3 + \ldots + a_n x^n = \\
    = a_0 + x \biggl(a_1 + x\Bigl(a_2 + \ldots
    + x\bigl(a_{n-1} + a_n x\bigr)\Bigr)\biggr).
\end{multline*}

\begin{example}
  Рассмотрим вычисление по схеме Горнера многочлена третьей степени.
  \[
  2 + 3x + 5 x^2 + 6 x^3 = 2 + x(3 + x(5 + 6 x)).
  \]

  В правой части используется 6 умножений (с учётом умножений,
  необходимых для возведения в степень) и 3 сложения. В левой части —
  3 умножения и 3 сложения.
\end{example}

Схему Горнера удобно реализовывать в виде цикла, выполняющего
вычисления начиная с самой внутренней скобки.

\Section{Степенные ряды}

\Term{Ряд!степенной}{Степенным рядом} называется бесконечная сумма
вида
\[
f(x) = \sum_{k=0}^{\infty} a_k x^k.
\]

Ряды играют важную роль в численных методах, так как многие функции
можно представить в виде степенных рядов. Часто это единственный
приемлемый способ вычислить значение какой-либо функции. Например,
функция $\sin x$ — трансцендентная, то есть её точные значения в
различных невозможно получить при помощи конечного числа
арифметических операций.

Многие трансцендентные функции допускают разложение в
\Term{Ряд!Тейлора}{ряд Тейлора}:
\[
  f(x + \Delta x) =
  f(x) + f'(x) \Delta x + \frac12 f''(x) \Delta x^2 + \ldots =
  \sum_{k=0}^{\infty}\frac{f^{(k)}(x)}{k!}\Delta x^{k}.
\]

Таким образом, если известны значения функции и её производных в
некоторой точке $x_{0},$ то можно получить её значение в точке,
удалённой от $x_{0}$ на расстояние $\Delta x$.

При расчётах на ЭВМ работать с бесконечными суммами невозможно,
поэтому вместо ряда Тейлора обычно пользуются следующей теоремой.

\begin{thm}[\TermS{Формула!Тейлора}{Формула Тейлора}]
  Если функция $f$ имеет $n+1$ производную на отрезке $[x - \Delta x,
    x + \Delta x]$, то
  \[
  \boxed{
  f(x + \Delta x) =
  f(x) + f'(x) \Delta x + \frac12 f''(x) \Delta x^2 + \ldots
  + \frac1{n!} f^{(n)}(x) \Delta x^n + R_{n+1}(x),
  }
  \]
  где
  \[
  R_{n+1} = \frac1{(n+1)!} f^{(n+1)}(x+\theta\Delta x)\Delta x^{n+1},
  \, 0<\theta<1.
  \]
\end{thm}

Величину $R_{n+1}(x)$ называют \Term{Остаточный член!в форме
  Лагранжа}{остаточным членом в форме Лагранжа}.

Часто интерес представляет не величина остаточного члена, а его
порядок. В этом случае принимают $R_{n+1}(x) = o(\Delta x^n).$ В этом
случае говорят об \Term{Остаточный член!в форме Пеано}{остаточном
  члене в форме Пеано}.

Эта формула чрезвычайно важна и часто используется при выводе
различных численных методов.

\Section{Сходимость степенного ряда}

\begin{defn}
Степенной ряд \Term{Сходимость!степенного рядя}{сходится}, если его
сумма равна конечному числу для некоторого аргумента $x$.
\end{defn}

Не все степенные ряды сходятся. Например, бесконечная сумма (так
называемый гармонический ряд)
\[
1+\frac{1}{2}+\frac{1}{3}+\frac{1}{4}+\ldots
\]
неограниченно возрастает.

Некоторые ряды сходятся условно, то есть лишь для некоторых значений
$x$.

\Section{Ускорение сходимости рядов}

При численных расчётах кроме самого факта сходимости важна также
\Term{Скорость сходимости}{скорость сходимости}. Она характеризуется
количеством слагаемых, которые необходимо взять, чтобы получить сумму
ряда с заданной точностью. Чем меньше слагаемых требуется, тем быстрее
сходится ряд.

Существуют различные методы ускореня сходимости рядов. Рассмотрим один из них,
основанный на \Term{Метод!Эйткена}{методе Эйткена}. Пусть $S_n$ —
сумма первых $n$ слагаемых (частичная сумма) некоторого ряда, тогда
ряд с частичной суммой
\[
S'_n = {\left(S_{n+1} - S_n\right)^2 \over S_{n-1} - 2 S_n + S_{n+1}}
\]
будет сходиться быстрее. Ускоренную последовательность можно снова
ускорить и так далее.

Этот метод хорошо работает с последовательностями частичных сумм
знакочередующихся рядов.

\begin{example}
Известно, что
\[
\frac{\pi}4 = 1 - \frac13 + \frac15 - \frac17 + \ldots.
\]

Для восьми слагаемых получаем значение $\pi \approx 3{,}017$.

Восемь слагаемых многократно ускоренной последовательности дадут 14
верных знаков числа $\pi$ после запятой.
\end{example}

\Practice

\Section{Вычисление рядов на ЭВМ}

???

Таким образом, алгоритм вычисления ряда
\[
\sum_{k=0}^{\infty}a_{k}x^{k}
\]
принимает следующий вид.


%\Procname{$\proc{Суммирование ряда}$}
%
%\zi Вход: $x, \varepsilon$
%\zi Выход: $sum$
%
%\li $sum \gets 0$
%\li $k \gets 0$
%\li \Repeat
%       	$a \gets a_k x^k$
%\li		$sum \gets sum + a$
%\li		$k \gets k + 1$
%\li \Until $|a| \geqslant \varepsilon$

\Tasks

Задачи:

Написать программу для приближённого вычисления значений заданной
функции при помощи разложения в ряд.

Сравнить результат с точным, вычислить абсолютную и относительную
погрешности.

Определить скорость сходимости.

Необходимо запрограммировать и исследовать следующие методы:

вычисление функций при помощи разложения в ряд (например, Тейлора).

При желании можно также реализовать:

какой-либо из ускорителей, например, ускоритель Эйлера.

В результатах должны быть представлены:
\begin{itemize}
\item точки, для которых вычислялась функция (4-5 штук);
\item задаваемые погрешности для критерия прекращения расчётов;
\item точные значения функций в выбранных точках;
\item вычисленные значения функции;
\item абсолютная и относительная погрешности;
\item количество итераций, потребовавшихся для получения результата.
\end{itemize}

\begin{center}
  \begin{tabular}{|c|c|c|c|c|c|c|}
    \hline 
    № п/п & $x_{0}$ & $\varepsilon$ & $y$ & $y^{*}$ & $\Delta y$ & $\delta y$\tabularnewline
    \hline 
    \hline 
    &  &  &  &  &  & \tabularnewline
    \hline 
  \end{tabular}
  \par
\end{center}
  
В заключительной части нужно сделать выводы, дающие ответ на следующие
вопросы:
\begin{itemize}
\item Насколько разложение в ряд подходит для вычисления значений
  специальных функций?
\item Велика ли погрешность?
\item От чего она зависит?
\item Как уменьшить погрешность вычислений?
\item От чего зависит скорость сходимости ряда? 
\end{itemize}

\begin{table}
  \begin{centering}
    \begin{tabular}{|c|c|c|c|}
      \hline № & $f(x)$ & Разложение в ряд & Условие сходимости\\
      \hline
      \hline
      0\vphantom{$\Biggl(\Biggr)$} & $\sin x$ & $\sum_{k=1}^{\infty}(-1)^{k-1}\frac{x^{2k-1}}{(2k-1)!}$ &
      $x\in\mathbb{R}$ \\
      \hline
      1\vphantom{$\Biggl(\Biggr)$} &
      $\cos x$ & $\sum_{k=0}^{\infty}(-1)^{k}\frac{x^{2k}}{(2k)!}$ &
      $x\in\mathbb{R}$\tabularnewline \hline 2\vphantom{$\Biggl(\Biggr)$} &
      $\ln(1+x)$ & $\sum_{k=1}^{\infty}(-1)^{k-1}\frac{x^{k}}{k}$ &
      $x\in(-1,1]$\tabularnewline \hline 3\vphantom{$\Biggl(\Biggr)$} &
        $\mathrm{arctg}\, x$ &
        $\sum_{k=1}^{\infty}(-1)^{k-1}\frac{x^{2k-1}}{2k-1}$ &
        $x\in[-1,1]$\tabularnewline \hline 4\vphantom{$\Biggl(\Biggr)$} &
        $\mathrm{sh}\, x=\frac{e^{x}-e^{-x}}{2}$ &
        $\sum_{k=1}^{\infty}\frac{x^{2k-1}}{(2k-1)!}$ &
        $x\in\mathbb{R}$\tabularnewline \hline 5\vphantom{$\Biggl(\Biggr)$}
        & $\mathrm{ch}\, x=\frac{e^{x}+e^{-x}}{2}$ &
        $\sum_{k=0}^{\infty}\frac{x^{2k}}{(2k)!}$ &
        $x\in\mathbb{R}$\tabularnewline \hline 6\vphantom{$\Biggl(\Biggr)$}
        & $\ln\frac{1+x}{1-x}$ & $2\sum_{k=0}^{\infty}\frac{x^{2k-1}}{2k-1}$
        & $x\in(-1,1)$\tabularnewline \hline 7\vphantom{$\Biggl(\Biggr)$} &
        $\frac{1}{1+x}$ & $\sum_{k=0}^{\infty}(-1)^{k}x^{k}$ &
        $x\in(-1,1)$\tabularnewline \hline 8\vphantom{$\Biggl(\Biggr)$} &
        $\frac{1}{(1-x)^{2}}$ & $\sum_{k=0}^{\infty}(k+1)x^{k}$ &
        $x\in(-1,1)$\tabularnewline \hline 9\vphantom{$\Biggl(\Biggr)$} &
        $e^{x}$ & $\sum_{k=0}^{\infty}\frac{x^{k}}{k!}$ &
        $x\in\mathbb{R}$\tabularnewline \hline
    \end{tabular}
    \par\end{centering}

    \caption{Варианты заданий}
\end{table}

\Questions

