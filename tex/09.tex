\Lab{Решение дифференциальных уравнений в частных производных}

\Aim{Исследовать численные методы решения дифференциальных уравнений в
частных производных с помощью электронных вычислительных машин.}

\Theory

\Section{Дискретизация}
В качестве примера дифференциального уравнения в частных производных
будем рассматривать хорошо известное \emph{уравнение переноса} (или
волновое уравнение):
\begin{equation}
\frac{\partial u}
{\partial t}+a\frac{\partial u}{\partial x}=0,\;-\infty<x<\infty.
\label{eq:transition}
\end{equation}


Это одномерное уравнение, так как неизвестная функция $u(x,t)$ зависит
только от одной пространственной переменной $x$. Так как функция
$u(x,t)$ зависит также от временн\'{о}й переменой $t$, то это уравнение
нестационарное.

Оно описывает процесс передачи некоторого сигнала, описываемого функцией
$u$ в фиксированный момент времени вдоль оси $x$ с некоторой скоростью
$a=\mathrm{const}$.

Для того, чтобы решить это уравнение, необходимо задать начальную
форму передаваемого сигнала в некоторый начальный момент времени $t_{0}$.
Обычно выбирается $t_{0}=0.$ Состояние системы в начальный момент
времени называется \emph{начальными условиями}. В данном случае они
имеют вид:
\begin{equation}
u(x,t)|_{t=t_{0}}=u_{0}(x).\label{eq:initial_cond}
\end{equation}


Математическая задача, описываемая начальными условиями \eqref{eq:initial_cond}
и уравнением, описывающем эволюцию системы, \eqref{eq:transition}
называют задачей с начальными данными или \emph{задачей Коши}.

Задача Коши описывает взаимосвязь между непрерывными величинами $u,$$x$
и $t$. К сожалению, решение дифференциальных уравнени в частных производных
чрезвычайно сложная задача, поэтому как правило требуется привлечение
численных методов и ЭВМ. В памяти ЭВМ нет возможности хранить всё
бесконечное количество значений, присутствующих в задаче, поэтому
применяют \emph{дискретизацию} \textemdash{} переход от непрерывных
величин к их значениям в некоторых дискретных точках. При этом, конечно,
происходит потеря части информации, поэтому в численных решениях как
правило присутствуют погрешности. Однако, если дискретные точки располагаются
достаточно близко, дискретизированная задача может оказаться достаточно
хорошо соответствующей исходной задаче. Исследование возникающих при
дискретизации погрешностей и адекватности численного решения \textemdash{}
достаточно сложный вопрос, который будет рассматриваться отдельно.

Для дискретизации пространственной переменной $x$ введём бесконечное
множество точек
\[
\{\dots,-2h,-h,0,h,2h,\dots\},
\]
где $h>0,\, h=\mathrm{const}$. Это множество задаёт на оси $x$ набор
равноудалённых точек с шагом $h$. Это множество точек называется
\emph{расчётной сеткой} на оси $x$ или пространственной сеткой. Элементы
множества \textemdash{} \emph{узлы сетки}. Так как узлы сетки в данном
случае равноудалены, то сетка называется \emph{равномерной}. На практике
используются и неравномерные сетки.

Пусть численное решение ищется некотором конечном временн\'{о}м интервале
$0<t<T,$ где $T>0$ \textemdash{} конечный момент времени. Аналогично
можно ввести временн\'{у}ю расчётную сетку 
\[
\{0=t_{0},t_{1},t_{2},\dots,t_{N}=T\}.
\]
 

Эта сетка также может быть равномерной. Обычно шаг временн\'{о}й сетки
обозначают $\tau.$

Для удобства вместо указания конкретных значений переменных можно
использовать их номера в сетке, которые определяются соотношениями
\begin{eqnarray*}
x_{j} & = & jh,\\
t_{n} & = & n\tau,
\end{eqnarray*}
где $-\infty<j<\infty$ и $0<n<N$ \textemdash{} номера узлов.

Можно ввести равномерную сетку $G_{h,\tau}$ на плоскости $(x,t)$
как множество точек пересечения прямых линий, задаваемых сетками для
каждой из переменных. Иначе говоря, $G_{h,\tau}$ \textemdash{} декартово
произведение пространственной и временн\'{о}й сеток.

Дискретизируем функцию $u$. Для этого будем рассматривать только
её значения в узлах пространственной и временн\'{о}й сеток. Для краткости
записи обычно используют следующее обозначение:
\[
u_{j}^{n}\equiv u(x_{j},t_{n}).
\]
Индекс соответствующий времени всегда пишется внизу, а индексы, соответствующие
пространственным переменным \textemdash{} внизу.

Функция $u_{j}^{n},$ определённая только в узлах $G_{h,\tau},$ называется
\emph{сеточной функцией}.

Начальные условия задают только часть значений $u_{j}^{n}.$ Задача
поиска численного решения заключается в том, чтобы найти все значения
сеточной функции в узлах выбранной сетки.

Для решения уравнения также необходимо провести дискретизацию операторов
дифференцирования, так как они неприменимы к сеточным функциям. В
численных методах операторы дифференцирования аппроксимируются разделёнными
разностями, которые можно получить из разложения в ряд Тейлора функции
$u.$
\begin{eqnarray*}
u(x_{j}+h,t_{n}) & = & u(x_{j},t_{n})+hu'(x_{j},t_{n})+\frac{h^{2}}{2}u''(x_{j},t_{n})+O(h^{3}),\\
u(x_{j}-h,t_{n}) & = & u(x_{j},t_{n})-hu'(x_{j},t_{n})+\frac{h^{2}}{2}u''(x_{j},t_{n})-O(h^{3}).
\end{eqnarray*}


Из первого выражения получаем формулу, имеющее название \guillemotleft{}разность
вперёд\guillemotright{}:
\[
u'(x_{j},t_{n})=\frac{u(x_{j}+h,t_{n})-u(x_{j},t_{n})}{h}+O(h)=\frac{u_{j+1}^{n}-u_{j}^{n}}{h}+O(h).
\]


Из второго выражения получаем \guillemotleft{}разность назад\guillemotright{}:
\[
u'(x_{j},t_{n})=\frac{u_{j}^{n}-u_{j-1}^{n}}{h}+O(h).
\]

Рассматривая полуразность исходных выражений можно получить формулу
для \guillemotleft{}центральной разности\guillemotright{}:
\[
u'(x_{j},t_{n})=\frac{u_{j+1}^{n}-u_{j-1}^{n}}{h}+O(h^{2}).
\]


Существуют и другие аппроксимации дифференциальных операторов.

Аналогично провести дискретизация оператора дифференцирования по времени.


\Section{Разностная задача Коши}

Простейший метод решения задачи \eqref{eq:transition}, \eqref{eq:initial_cond}
\textemdash{} так называемый метод конечных разностей. Его идея заключается
в простой замене всех величин их дискретизированными аналогами.

Заменим производную по пространству в узле $(x_{j},t_{n})$ разностью
назад, а по времени \textemdash{}~разностью вперёд пренебрегая членами
высоких порядков:
\begin{eqnarray*}
\frac{\partial u}{\partial x} & \approx & \frac{u_{j}^{n}-u_{j-1}^{n}}{h},\\
\frac{\partial u}{\partial t} & \approx & \frac{u_{j}^{n+1}-u_{j}^{n}}{\tau}.
\end{eqnarray*}


Подставляя полученные выражения в \eqref{eq:transition}, получаем
алгебраическое уравнение
\begin{equation}
\frac{u_{j}^{n+1}-u_{j}^{n}}{\tau}+a\frac{u_{j}^{n}-u_{j-1}^{n}}{h}=0.\label{eq:diff_scheme}
\end{equation}


Можно также провести дискретизацию начальных условий:
\begin{equation}
u_{j}^{0}=u_{0}(x_{j}).\label{eq:diff_initials}
\end{equation}


Уравнение \eqref{eq:diff_scheme} называют \emph{конечно-разностным
уравнением} или \emph{разностной схемой}. Задача, задаваемая выражениями
\eqref{eq:diff_scheme} , \eqref{eq:diff_initials} называется \emph{конечно-разностной
задачей Коши}.

Начальные условия \eqref{eq:diff_initials} позволяют найти значения
на так называемом нулевом слое по времени \textemdash{} $u_{j}^{0}.$
Рассмотрим применение разностной схемы для поиска остальных значений.
Для этого преобразуем выражение \eqref{eq:diff_scheme}.

\[
u_{j}^{n+1}=u_{j}^{n}-\underbrace{\frac{a\tau}{h}}_{\equiv\kappa}(u_{j}^{n}-u_{j-1}^{n}).
\]


Полученное выражение позволяет получить любое значение на временн\'{о}м
слое $n+1$ зная значения на слое $n.$ В частности, опираясь на начальные
условия можно получить значения на следующем, первом слое по времени:
\[
u_{j}^{1}=u_{j}^{0}-\kappa(u_{j}^{0}-u_{j-1}^{0}).
\]


Аналогично вычисляются значения на последующих слоях.

Величина $\kappa$ называется числом Куранта и играет важную роль
при исследовании разностных схем.

Рассмотренная схема называется \guillemotleft{}левый уголок\guillemotright{}.
Название обусловлено фотрмой шаблона разностной схемы \textemdash{}
диаграммы, на которой отмечены участвующие в вычислениях узлы, соединённые
линиями.

Разностная схема \eqref{eq:diff_scheme} называется \emph{явной},
так как содержит единственное значение на самом верхнем слое, которое
явно выражается через значения предыдущих слоёв. В схеме присутствуют
значения из двух слоёв, поэтому она также называется \emph{двухслойной}.

На практике, однако, решить это уравнение рассмотренным методом не
удастся, так как ось $x$ предполагается бесконечной в обоих направлениях,
что потребует бесконечного количества памяти для хранения значений
сеточной функции.

Можно ограничить пространственную сетку:
\[
\{0,h,2h,3,h,\dots Mh\}.
\]


Однако, в этом случае возникает проблем с вычислением значений сеточной
функции на левой границе, так как величина $u_{0}^{n+1}$ зависит
от $u_{-1}^{n},$ значение которой не определено.

Для устранения этой проблемы необходимо задать кроме начальных также
и \emph{граничные условия}, позволяющие без использования разностной
сетки вычислять значения на левом крае. Это можно сделать, определив
некоторую функцию на границе:
\begin{equation}
u_{0}^{n}=g(t_{n}).\label{eq:diff_bound}
\end{equation}


Существуют и другие способы задания граничных условий.

Выражения \eqref{eq:diff_scheme}, \eqref{eq:diff_initials} и\eqref{eq:diff_bound}
задают так называемую \emph{разностную начально-краевую задачу}.

\Practice

\Tasks

\Questions
