\newlength{\UDCLength}
\setlength{\UDCLength}{\maxof{\widthof{УДК}}{\widthof{ББК}}}
\noindent\makebox[\UDCLength][l]{УДК} \UDC\\
\noindent\makebox[\UDCLength][l]{ББК} \BBK\\
\noindent\makebox[\UDCLength][l]{} \AS

\vfill

\noindent{\addfontfeature{LetterSpace=30}Рецензенты}:

\noindent\textbf{???}, ???

\noindent\textbf{???}, ???

\vfill

\newlength{\ASShiftLength}
\setlength{\ASShiftLength}{-0.5em-\widthof{\AS}}
\noindent {\hspace{\ASShiftLength}\AS}\hfill%
\begin{minipage}[t]{1\columnwidth}%
  \noindent\hspace{2em}\textbf{\AuthorI}
  
  \noindent\hspace{2em}\Title. \SubTitle: \PubType.
  — Тирасполь,~\Year. — \pageref{LASTPAGE}~с. %%% FIXME Количество страниц
  \medskip{}
  
  \hspace{2em}{\small Лабораторный практикум состоит из девяти работ,
    охватывающих основные разделы численных методов. В каждой работе
    приводится краткое изложение необходимых для выполнения
    теоретических сведений. Для выполнения заданий рекомендуется
    использовать какую-либо среду матричных вычислений: MatLab,
    Octave или их аналоги.}
  
  \hspace{2em}{\small Практикум предназначен для студентов, обучающихся по
    направлению «Информатика».}
\end{minipage}

\noindent
\begin{flushright}
  \begin{minipage}[t]{0.5\columnwidth}
    \noindent\makebox[\UDCLength][l]{УДК} \UDC\\
    \noindent\makebox[\UDCLength][l]{ББК} \BBK
  \end{minipage}
\end{flushright}

\vfill{}

\begin{center}
  Утверждено Научно-методическим советом ПГУ~им.~Т.~Г.~Шевченко
\end{center}

\vfill{}

\noindent {\small © \AuthorI, \Year.}

\thispagestyle{empty}
\newpage
