\Lab{Интерполяция}

\Aim{Исследовать численные методы интерполяции при помощи электронных
  вычислительных машин.}

\Theory

\Section{Понятие интерполяции}

\emph{Интерполяция} --- это способ приближенного или точного
нахождения какой-либо величины по известным отдельным значениям этой
же или других величин, связанных с ней. Наибольшее значение в
вычислительной математике имеет задача построения способов
интерполяции функций.

Иногда возникает задача приближенной замены или
\emph{аппроксимации}\index{Аппроксимация}\emph{ }некоторой функции
другими функциями, которые легче вычислить. Кроме того, на практике
случайные ошибки в значениях функции сильно искажают интерполяционные
полиномы высоких степеней, поэтому при наличии случайных ошибок часто
предпочитают применять сглаживающую апроксимацию такими многочленами
или рациональными дробями, которые минимизируют какой-либо критерий
близости, например, взвешенную среднеквадратическю ошибку
аппроксимации.

Заметим, что разложение в ряд Тейлора аппроксимирует аналитическую
функцию только в непосредственной близости от одной выбранной точки и
потому редко применяется.

Задача интерполяции возникает, например, в случае, когда известны
результаты измерений $y_{k}=y(x_{k})$ некоторой физической величины
$y(x)$ в точках $x_{k},k=0,1,\dots,n$ и требуется определить ее
значения в других точках.

К интерполяции приходится иногда прибегать и в том случае, когда для
функции $y(x)$ известно и аналитическое представление, с помощью
которого можно вычислять ее значения для любого значения $x$ из
области ее определения, но вычисление каждого значения сопряжено с
большим объемом вычислений. Для уменьшения объема вычислений вычисляют
несколько значений $y(x)$ и по ним вычисляют приближенные значения в
остальных точках.

Интерполяционная формула сопоставляет с функцией $y(x)$ функцию
известного класса
\[
Y(x)\equiv Y(x;\alpha_{0},\alpha_{1},\dots,\alpha_{n}),
\]
зависящую от $n+1$ параметров $\alpha_{j}$, выбранных так, чтобы
значения значения $Y(x)$ совпадали со значениями $y(x)$ для данного
множества значений аргумента $\{x_{k}\}$ \emph{--- узлов
  интерполяции}:
\[
Y(x_{k})=y(x_{x})=y_{k}.
\]


Пусть на отрезке $[a,b]$ задана система функций
$\{\varphi_{k}(x)\}_{k=0}^{n}$ и введена сетка
$\{x_{k}\}_{k=0}^{n}\subset[a,b]$. Образуем линейную комбинацию
\[
\varphi(x)=c_{0}\varphi_{0}(x)+c_{1}\varphi_{1}(x)+\dots+c_{n}\varphi_{n}(x)
\]
с действительными коэффициентами $c_{0},c_{1},\dots,c_{n}$. Задача
интерполирования функции $f(x)$ системой
$\{\varphi_{k}(x)\}_{k=0}^{n}$ на сетке
$\{x_{i}\}_{i=0}^{n}\subset[a,b]$ состоит в нахождении коэффициентов,
для которых
\[
\varphi(x_{i})=f(x_{i}),\, i=0,1,\dots,n.
\]


Интерполяция полиномами --- частный случай рассмотренной
задачи. Запишем условие равенства более подробно:
\begin{eqnarray*}
c_{0}\varphi_{0}(x_{0})+c_{1}\varphi_{1}(x_{0})+\dots+c_{n}\varphi_{n}(x_{0})
& = &
f(x_{0}),\\ c_{0}\varphi_{0}(x_{1})+c_{1}\varphi_{1}(x_{1})+\dots+c_{n}\varphi_{n}(x_{1})
& = & f(x_{1}),\\ &
\vdots\\ c_{0}\varphi_{0}(x_{n})+c_{1}\varphi_{1}(x_{n})+\dots+c_{n}\varphi_{n}(x_{n})
& = & f(x_{n}).
\end{eqnarray*}

Чтобы эта система имела единственное решение относительно
коэффициентов $\{c_{k}\}$, необходимо и достаточно, чтобы определитель
матрицы
\[
A=\left|\begin{array}{cccc} \varphi_{0}(x_{0}) & \varphi_{1}(x_{0}) &
\cdots & \varphi_{n}(x_{0})\\ \varphi_{0}(x_{1}) & \varphi_{1}(x_{1})
& \cdots & \varphi_{n}(x_{1})\\ \vdots & \vdots & \ddots &
\vdots\\ \varphi_{0}(x_{n}) & \varphi_{1}(x_{n}) & \cdots &
\varphi_{n}(x_{n})
\end{array}\right|
\]
был отличен от нуля. Более того, поскольку узлы $\{x_{i}\}$
расположены произвольно, необходимо потребовать
\[
\det A\neq0
\]
для любой интерполяционной сетки.

Система функций $\{\varphi_{k}(x)\}_{k=0}^{n}$ называется
\emph{системой Чебышёва} на $[a,b]$, если определитель $\det A\neq0$
при любом расположении несовпадающих узлов
$\{x_{i}\}_{i=0}^{n}\subset[a,b]$.  Функция $\varphi(x)$ называется
\emph{обобщенным интерполяционным полиномом} по системе
$\{x_{i}\}_{i=0}^{n}$.

Например, система алгебраических многочленов $\varphi_{k}(x)=x^{k}$
является чебышёвской системой на любом отрезке $[a,b]$.

\Section{Интерполяционная формула Лагранжа}

Среди всех методов интерполяции особое место занимает
\emph{интерполяция полиномами} (многочленами). Это связано со
следующими обстоятельствами:
\begin{enumerate}
\item Полиномы легко вычислять.
\item Множество полиномов плотно в пространстве непрерывных функций
  (аппроксимационная теорема Вейерштрасса), то есть любую непрерывную
  функцию можно сколь угодно точно приблизить
  полиномами.
\end{enumerate}

\begin{thm}[\emph{Вейерштрасс}]
  Для любой непрерывной на отрезке $[a,b]$ функции f и для любого
  $\varepsilon>0$ существует такое $n\in\mathbb{N}$ и такой полином
  $p_{n}(x)$ степени $n$, что равномерная норма (норма Чебышёва)
  меньше $\varepsilon$:
\[
\left\Vert f-p_{n}\right\Vert
_{C_{[a,b]}}=\max_{x\in[a,b]}|f(x)-p_{n}(x)|<\varepsilon.
\]

\end{thm}
Пусть на отрезке $[a,b]$ заданы несовпадающие точки
$x_{k},k=0,1,\dots,n$ в которых известны значения функции
$f(x)$. Тогда задача интерполяции алгебраическими многочленами состоит
в том, чтобы построить многочлен
\[
L_{n}(x)=a_{0}+a_{1}x+\dots+a_{n}x^{n}
\]
степени $n$, значения которого в заданных точках $x_{k}$ совпадают со
значениями функции $f$ в этих точках. Многочлен $L_{n}(x)$ в этом
случае назвается интерполяционным многочленом.

Для любой непрерывной функции $f(x)$ эта задача имеет единственное
решение. Его можно записать в различных формах, наиболее употребимыми
из которых являются формы Лагранжа и Ньютона.

Интерполяционная формула Лагранжа представляет собой многочлен в виде
комбинации
\[
L_{n}(x)=\sum_{k=0}^{n}c_{k}(x)f(x_{k})
\]
значений функции $f(x)$ в узлах интерполирования. $L_{n}(x)$ называют
\emph{интерполяционным полиномом Лагранжа}\index{Интерполяционный
  полином!Лагранжа}.

Найдем выражение для коэффициентов $c_{k}(x)$. Из условий
интерполирования
\[
\sum_{k=0}^{n}c_{k}(x_{i})f(x_{k})=f(x_{i}),
\]
значит
\[
c_{k}(x_{i})=\begin{cases} 0, & i\neq k,\\ 1, & i=k,\, i=0,1,\dots,n,
\end{cases}
\]
то есть каждая из функций $c_{k}(x)$ имеет не меньше $n$ нулей на
отрезке $[a,b]$.

Будем искать их в виде многочленов степени $n$, а именно в виде
\[
c_{k}(x)=\lambda_{k}(x-x_{0})(x-x_{1})\dots(x-x_{k-1})(x-x_{k+1})\dots(x-x_{n}).
\]


Из условия $c_{k}(x_{k})=1$ находим
\[
\lambda_{k}=\frac{1}{(x_{k}-x_{0})(x_{k}-x_{1})\dots(x_{k}-x_{k-1})(x_{k}-x_{k+1})\dots(x_{k}-x_{n})}.
\]


Для краткости обозначим $f(x_{k})$ через $f_{k}$. Итак, коэффициенты
вычисляются по формуле:
\begin{multline*}
c_{k}(x)=\\ =\frac{(x-x_{0})(x-x_{1})\dots(x-x_{k-1})(x-x_{k+1})\dots(x-x_{n})}{(x_{k}-x_{0})(x_{k}-x_{1})\dots(x_{k}-x_{k-1})(x_{k}-x_{k+1})\dots(x_{k}-x_{n})}=\\ =\frac{\prod_{i\neq
    k}(x-x_{i})}{\prod_{i\neq k}(x_{k}-x_{i})}.
\end{multline*}
а интерполяционный многочлен принимает вид:
\[
L_{n}(x)=\sum_{k=0}^{n}f_{k}\prod_{i\neq k}\frac{x-x_{i}}{x_{k}-x_{i}}.
\]


При $n=1$ получаем формулу для \emph{линейной интерполяции}:
\[
L_{1}(x)=\frac{x-x_{1}}{x_{0}-x_{1}}f(x_{0})+\frac{x-x_{0}}{x_{1}-x_{0}}f(x_{1})=\frac{f_{1}-f_{0}}{x_{1}-x_{0}}(x-x_{0}),
\]
по которой приближенные значения считаются лежащими на отрезке,
соединяющем два соседних узла.

\Section{Интерполяционная формула Ньютона}

Интерполяционный полином степени $n$, проходящий через заданные $n+1$
точку единственен. Однако, запись его в форме Лагранжа для некоторых
задач может оказаться неудобной, так как для заданного набора
интерполяционных узлов все лагранжевы полиномы имеют одну и ту же
степень --- $n$.  В частности, при добавлении нового интерполяционного
узла нельзя воспользоваться полученными ранее полиномами, и для более
высокой степени их приходится строить заново.

\emph{Разделенной разностью}\index{Разделенная разность} назовем
величину, определяюмую рекуррентными соотношениями:
\begin{eqnarray*}
\Delta_{1}(x_{0},x_{1}) & \equiv & \frac{f_{1}-f_{0}}{x_{1}-x_{0}},\\
\Delta_{k}(x_{0},x_{1},\dots,x_{k}) & \equiv & \frac{\Delta_{k-1}(x_{1},x_{2},\dots,x_{k})-\Delta_{k-1}(x_{0},x_{1},\dots,x_{k-1})}{x_{k}-x_{0}}.
\end{eqnarray*}


Нетрудно видеть, что разделенные разности имеют размерности
соответствующих производных.
\begin{table}
\noindent \begin{centering}
\begin{tabular}{|c|c|c|c|c|}
\hline 
$x_{0}$ &  &  &  & \tabularnewline
\hline 
 & $\Delta_{1}(x_{0},x_{1})$ &  &  & \tabularnewline
\hline 
$x_{1}$ &  & $\Delta_{2}(x_{0},x_{1},x_{2})$ &  & \tabularnewline
\hline 
 & $\Delta_{1}(x_{1},x_{2})$ &  & $\Delta_{3}(x_{0},x_{1},x_{2},x_{3})$ & \tabularnewline
\hline 
$x_{2}$ &  & $\Delta_{2}(x_{1},x_{2},x_{3})$ &  & $\cdots$\tabularnewline
\hline 
 & $\Delta_{1}(x_{2},x_{3})$ &  & $\Delta_{3}(x_{1},x_{2},x_{3},x_{4})$ & \tabularnewline
\hline 
$x_{3}$ &  & $\Delta_{2}(x_{2},x_{3},x_{4})$ &  & \tabularnewline
\hline 
$\vdots$ & $\vdots$ & $\vdots$ & $\vdots$ & $\ddots$\tabularnewline
\hline 
\end{tabular}
\par\end{centering}

\caption{Вычисление разделенных разностей}
\label{tab:divdiff}
\end{table}

В таблице \ref{tab:divdiff} показан процесс вычисления разделенных
разностей для нескольких первых порядков.

\emph{Интерполяционным полиномом Ньютона}\index{Интерполяционный
  полином!Ньютона} называется многочлен вида
\begin{multline*}
P_{n}(x)=f_{0}+(x-x_{0})\Delta_{1}(x_{0},x_{1})+\\
+(x-x_{0})(x-x_{1})\Delta_{2}(x_{0},x_{1},x_{2})+\dots+\\
+(x-x_{0})(x-x_{1})\dots(x-x_{n-1})\Delta_{n}(x_{0},x_{1},\dots,x_{n}).
\end{multline*}


Покажем, что он решает задачу интерполяции. Для этого рассмотрим
разделенные разности некоторого интерполяционного полинома $P_{n}(x)$,
в которых в качестве первых аргументов выступает сама переменная $x$,
а остальными являются точки интерполяции. Степень этого полинома равна
$n$. Разность $P_{n}(x)-P_{n}(x_{0})=P_{n}(x)-f_{0}$ обращается в ноль
в точке $x_{0}$ и, следовательно, делится на $x-x_{0}$.

Итак, разделенная разность
\[
\delta_{1}(x_{0},x)=\frac{P_{n}(x)-f_{0}}{x-x_{0}},
\]
является полиномом степени $n-1$. Аналогично, вторая разность
$\delta_{2}(x_{0},x_{1},x)=\frac{\delta_{1}(x_{1},x)-\delta_{1}(x_{0,}x_{1})}{x-x_{1}}$
--- полином по $x$ степени $n-2$ и так далее до разности $p_{01\dots
  n}(x)$, которая уже не зависит от $x$ и является
константой. Разности более высокого порядка будут равны нулю.

Таким образом, <<разворачивая>> выражения начиная с первого, получаем
\begin{multline*}
P_{n}(x)=f_{0}+(x-x_{0})\delta_{1}(x_{0},x)=\\
=f_{0}+(x-x_{0})\left(\delta_{1}(x_{0},x_{1})-(x-x_{1})\delta_{2}(x_{0},x_{1},x)\right)=\dots=\\
=f_{0}+(x-x_{0})\delta_{1}(x_{0},x_{1})+\dots+\\
+(x-x_{0})(x-x_{1})\dots(x-x_{n})\delta_{n}(x_{0},x_{1},\dots,x_{n}).
\end{multline*}


Учитывая, что значения многочлена по условию совпадают в узлах
интерполяции со значениями функции, а значит
$\delta_{k}=\Delta_{k}$. В итоге получили форму Ньютона для
интерполяционного полинома.

Интерполяционную формулу Ньютона удобнее применять в том случае, когда
интерполируется одна и та же функция, но число узлов интерполяции
постепенно увеличивается. Если узлы интерполяции фиксированы и
интерполируется не одна, а несколько функций, то удобнее пользоваться
формулой Лагранжа.

\Section{Погрешность интерполяции}

Заменяя функцию $f(x)$ интерполяционным многочленом $L_{n}(x)$ мы
допускаем погрешность
\[
r_{n}(x)=f(x)-L_{n}(x),
\]
которая называется \emph{погрешностью интерполяции}. Очевидно, что в
интерполяционных узлах она равна нулю. Если не накладывать на функцию
никаких ограничений, но сказать определенного ничего нельзя, но при
достаточной гладкости функции можно получить некоторые оценки
погрешности.

Пусть $f$ --- непрерывная на $[a,b]$ функция и $L_{n}$ ---
интерполяционный полином, удовлетворяющие одному и тому же множеству
из $n+1$ интерполяционного узла $\{x_{i}\}$, тогда для любой точки
$x\in[a,b]$ существует такая точка $\xi(x)$, что
\[
r_{n}(x)=\frac{f^{(n+1)}\left(\xi(x)\right)}{(n+1)!}N_{n+1}(x),
\]
где $N_{n+1}(x)=(x-x_{0})(x-x_{1})\dots(x-x_{n})$.

При произвольном расположении узлов оценить $|N_{n+1}(x)|$ довольно
сложно. Для равномерной сетки ситуация выглядит проще. Проведем грубую
оценку. Пусть $x\in[x_{k-1},x_{k}]$, тогда
\[
|x_{0}-x|\leqslant kh,\,|x_{1}-x|\leqslant(k-1)h,\dots,\,|x_{k-1}-x|\leqslant h,
\]
\[
|x_{k}-x|\leqslant h,\,|x_{k+1}-x|\leqslant2h,\dots,\,|x_{n}-x|\leqslant(n-k+1)h,
\]
откуда получаем $|N_{n+1}|\leqslant h^{n+1}k!(n-k+1)!$. Отсюда
\[
\|r_{n}\|\leqslant\|f^{(n+1)}\|\underset{\frac{1}{C_{n+1}^{k}}}{\underbrace{\frac{k!(n-k+1)!}{(n+1)!}}}h^{n+1},
\]
то есть $|f-L_{n}|=O(h^{n+1})$. Говорят, что интерполяционный многочлен
$L_{n}$ имеет погрешность $O(h^{n+1})$.

Поскольку полиномы Лагранжа и Ньютона отличаются только формой записи,
полученный оценки верны для них обоих.

Заметим, что можно подобрать узлы так, чтобы величина
$\max|N_{n+1}(x)|$ была меньше чем у любого другого полинома этой же
степени с единичным старшим коэффициентом. Такие наименее
отклоняющиеся от нуля полиномы --- многочлены Чебышёва.

Для увеличения точности можно использовать следующие методы построения
полинома:
\begin{enumerate}
\item Уменьшение расстояния между узлами при постоянной степени интерполяционного
полинома. Но при этом сужается область хорошей интерполяции.
\item Оптимизация размещения узлов. Обычно это означает выбор в
  качестве узлов корней многочленов Чебышёва.
\item Увеличение числа узлов, что означает повышение степени полинома.
\end{enumerate}

Будем говорить, что интерполяционный процесс $L_{n}(x)$ \emph{сходится
  равномерно} на $[a,b]$, когда
\[
\max_{x\in[a,b]}|f(x)-L_{n}(x)|\underset{n\rightarrow\infty}{\longrightarrow}0.
\]


Следует иметь в виду следующую теорему.
\begin{thm}[\emph{Фабер}]
\emph{}Для любой последовательности сеток найдется непрерывная на
$[a,b]$ функция $f(x)$ такая, что последовательность интерполяционных
полиномов $L_{n}(x)$ не сходится к $f(x)$ равномерно на $[a,b]$.
\end{thm}
Так, даже для функции $g(x)=|x|$ на равномерной сетке значения интерполяционного
полинома неограниченно возрастают между узлами в окрестностях точек
$-1$ и $1$.

Для заданной непрерывной функции $f(x)$ можно добиться сходимости
за счет выбора расположения узлов интерполяции.
\begin{thm}[\emph{Марцинкевич}]
\emph{}Если $f(x)$ непрерывна на $[a,b]$, то найдется такая
последовательность сеток, для которой соответствующий интерполяционный
процесс будет сходиться равномерно на $[a,b]$.
\end{thm}

\Section{Наилучшее приближение}

Пусть значения аппроксимируемой функции $f(x)$ и чебышёвской системы
аппроксимирующих функций $\{\varphi_{k}(x)\}_{k=0}^{n}$ известны в
узлах интерполяционной сетки $\{x_{k}\}_{k=0}^{m}\subset[a,b]$.  В
случае $m>n$ задача интерполирования становится переопределенной.  В
этом случае рассматривают задачу о \emph{наилучшем
  приближении}\index{Приближение!наилучшее}.

Рассматривают разности между значениями обобщенного многочлена и
интерполируемой функции в узлах интерполяционной сетки:
\[
r_{k}=\varphi(x_{k})-f(x_{k}),\, k=0,1,\dots,m.
\]

Для вектора погрешностей
$\mathbf{r}=\left(r_{0},r_{1},\dots,r_{m}\right)$ можно ввести ту или
иную норму. Например:
\begin{equation}
\|\mathbf{r}\|=\sqrt{\sum_{k=1}^{m}r_{k}^{2}}=\sqrt{\sum_{k=1}^{m}\Bigl(\varphi(x_{k})-f(x_{k})\Bigr)^{2}}\label{eq:meansqnorm}
\end{equation}
или
\begin{equation}
\|\mathbf{r}\|=\max_{0\leqslant k\leqslant m}|r_{k}|=\max_{0\leqslant k\leqslant m}|\varphi(x_{k})-f(x_{k})|.\label{eq:ravnnorm}
\end{equation}


Задача о наилучшем приближении функции $f(x)$ состоит в в нахождении
коэффициентов $c_{0},c_{1},\dots,c_{n}$, минимизирующих норму вектора
$\mathbf{r}$. В зависимости от выбора нормы получим различные задачи.
Так, норме \eqref{eq:meansqnorm} соответствует задача о наилучшем
среднеквадратическом приближении, а норме \eqref{eq:ravnnorm} ---
задача о наилучшем равномерном приближении функции.

Если $m=n$, то независимо от выбора нормы решение
$\mathbf{c}=(c_{0},c_{1},\dots,c_{n})$ задачи о наилучшем приближении
совпадает с решением задачи интерполирования.  Действительно, в этом
случае требование $\|\mathbf{r}\|=0$ приводит к условиям
\[
\varphi(x_{k})=f(x_{k}),\, k=0,1,\dots,n.
\]

\Section{Сплайн-интерполяция}

Интерполяция на больших интервалах, то есть с относительно большим
количеством узлов, имеет дополнительные трудности. С одной стороны,
точность при больших расстояниях между узловыми точками очень мала, а
с другой стороны, интерполяционные многочлены высокого порядка на
концах интервала значительно колеблются, что существенно искажает
поведение функции. Это становится особенно важным при последующем
дифференцировании. При решении подобных задач оказывает помощь:

\begin{enumerate}
\item \emph{Кусочная интерполяция}\index{Интерполяция!кусочная} более
  низкого порядка. При этом интерполяция осуществляется по небольшому
  числу узловых точек, а затем многочлены объединяют в общую
  интерполяционную функцию.
\item \emph{Интерполяция сплайнами}\index{Интерполяция!сплайнами}.
\end{enumerate}
Пусть задано, как и прежде, $n+1$ узловых точек. Функцию,
интерполирующую $f(x)$ на $[a,b]$ будем искать среди так называемых
сплайн-функций $S_{n}$ степени $n$.

Функция $S_{m}(x)$ называется
\emph{сплайн-функцией}\index{Сплайн-функция} степени $m$ на множестве
узлов $\{x_{k}\}_{k=0}^{n}$ если она и ее производные до порядка $n-1$
включительно непрерывны на отрезке $[a,b]$ и $S_{m}(x)$ --- многочлен
степени не больше $m$ при $x\in[x_{k-1},x_{k}]$.  Заметьте, что
сплайн-функция в общем случае не является многочленом.

Разница между степенью сплайна и порядком его высшей непрерывной производной
называется \emph{дефектом сплайна}\index{Сплайна дефект}.

При $m=3$ говорят о кубических сплайнах. Они наиболее распространены,
так как на практике достаточно обеспечить лишь непрерывность первой и
второй производных.

Параметры, задающие сплайны определяют, решая систему уравнений
(неизвестные --- коэффициенты полиномов, входящих в сплайн),
построенную на основе условий равенства значений в узлах значениям
функции, равенства производных смежных компонентов сплайна и каких-то
граничных условий на отрезке.

\Practice

\Tasks

Написать программу для вычисления интерполяционного многочлена в точке
для заданных значений функции в узлах интерполяционной сетки. Сравнить
результат с точным, вычислить абсолютную и относительную погрешности.

Необходимо запрограммировать и исследовать следующие методы:

интерполяционный полином Лагранжа.

При желании можно также реализовать:

интерполяционный полином Ньютона, сплайн-интерполяцию.

В результатах должно быть представлено:

исходные данные — точки интерполяционной сетки (желательно 4-5 точек)
и значения функции в них; точки, в которых вычислялся полином; значения
полинома и точные значения функции в них, абсолютные и относительные
погрешности в каждой из точек.

В заключительной части нужно сделать выводы, дающие ответ на следующие
вопросы:

Насколько интерполяция полиномами подходит для приближённого
вычисления значений функций?

Велика ли погрешность?

От чего она зависит?

Где погрешность самая низкая, а где самая высокая?

Как уменьшить погрешность вычислений?

\Questions
