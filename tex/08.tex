\section{Решение обыкновенных дифференциальных уравнений}

\Aim{Исследовать методы решения обыкновенных дифференциальных
  уравнений с помощью электронных вычислительных машин.}
